\documentclass[11pt]{beamer}
\usepackage[utf8]{inputenc}
%
\mode<presentation> {
\usetheme{CambridgeUS}
}
%
\setbeamercolor{frametitle}{fg=cyan!80!black}
\setbeamercolor{title}{fg=cyan!80!black}
\setbeamercolor{palette tertiary}{fg=black, bg=cyan}
\setbeamercolor{palette primary}{fg=black, bg=gray!30!white}
\setbeamercolor{palette secondary}{fg=black, bg=gray!20!white}
%
\setbeamertemplate{headline}
{
  \leavevmode%
  \hbox{%
%  %\begin{beamercolorbox}[wd=.5\paperwidth,ht=2.25ex,dp=1ex,right]{section in head/foot}%
%  %  \usebeamerfont{section in head/foot}\insertsectionhead\hspace*{2ex}
%  %\end{beamercolorbox}%
%  %\begin{beamercolorbox}[wd=.5\paperwidth,ht=2.25ex,dp=1ex,left]{subsection in head/foot}%
%  %  \usebeamerfont{subsection in head/foot}\hspace*{2ex}\insertsubsectionhead
%  %\end{beamercolorbox}
  }%
  \vskip0pt%
}
\setbeamertemplate{footline}{
  \leavevmode%
  \hbox{%
%  %\begin{beamercolorbox}[wd=.5\paperwidth,ht=2.25ex,dp=1ex,center]{author in head/foot}%
%  %  \usebeamerfont{author in head/foot}{Vipin @ R\"{a}tschLab}
%  %\end{beamercolorbox}%
%  %\begin{beamercolorbox}[wd=.5\paperwidth,ht=2.25ex,dp=1ex,center]{title in head/foot}%
%  %  \usebeamerfont{title in head/foot}\insertshorttitle
%  %\end{beamercolorbox}%
%  %\begin{beamercolorbox}[wd=.333333\paperwidth,ht=2.25ex,dp=1ex,right]{date in head/foot}%
%    %\usebeamerfont{date in head/foot}\insertshortdate{}\hspace*{2em}
%  %\end{beamercolorbox}
  }%
  \vskip0pt%
}
%
\usepackage{graphicx} 
\usepackage{booktabs}
\setbeamertemplate{navigation symbols}{}
\usepackage{listings}
\usepackage{color}

\definecolor{codecomment}{rgb}{152,0,0}
\definecolor{codegray}{rgb}{0.5,0.5,0.5}
\definecolor{codepurple}{rgb}{0.58,0,0.82}
\definecolor{backcolour}{rgb}{0.95,0.95,0.92}
\definecolor{codekeys}{rgb}{30,0,100}

\lstdefinestyle{mystyle}{
    backgroundcolor=\color{backcolour},   
    commentstyle=\color{codecomment},
    keywordstyle=\color{codekeys},
    numberstyle=\tiny\color{codegray},
    stringstyle=\color{codepurple},
    basicstyle=\footnotesize,
    breakatwhitespace=false,         
    breaklines=true,                 
    captionpos=b,                    
    keepspaces=true,                 
    numbers=left,                    
    numbersep=5pt,                  
    showspaces=false,                
    showstringspaces=false,
    showtabs=false,                  
    tabsize=4
}

\lstset{style=mystyle}

\newcommand{\thankyou}
{\begin{center} Would love to hear your experience! \\ gabow@cbio.mskcc.org \\ vipin@cbio.mskcc.org \end{center}}
%
\title[pyClass1]{Introduction to Programming with Python - Day 3}
\author{vipin@cbio.mskcc.org}
%
\institute[cBio@MSKCC]
{
    R{\"a}tsch Laboratory, Computational Biology Center\\
    Memorial Sloan Kettering Cancer Center\\
}
\date{11 November 2014}
%
\begin{document}
\maketitle
%
%% Continue from Data structures 
\begin{frame}[plain]
    \frametitle{List Comprehensions}
    \begin{itemize}
        \item[] List comprehensions provide a concise way to create lists. 
        %\pause
        \lstinputlisting[language=Python]{scripts/list_comprehension_1.py}
        %\pause
        \lstinputlisting[language=Python]{scripts/list_comprehension_2.py}
        %\pause
        \lstinputlisting[language=Python]{scripts/list_comprehension_3.py}
        %\pause
        \lstinputlisting[language=Python]{scripts/list_comprehension_4.py}
    \end{itemize}
\end{frame}
%
\begin{frame}[plain]
    \frametitle{del Statement}
    \begin{itemize}
        \item[] The \textbf{del} statement is to remove an item from a list given its index instead of its value.
        \newline
        %\pause
        \lstinputlisting[language=Python]{scripts/list_del.py}
    \end{itemize}
\end{frame}
%%
\begin{frame}[plain]
    \frametitle{Tuple - Another Data Type}
    \begin{itemize}
        \item[] Tuples are immutable, and usually contain an heterogeneous sequence of elements. 
        \newline
        %\pause
        \lstinputlisting[language=Python]{scripts/tuple.py}
    \end{itemize}
\end{frame}
%%
\begin{frame}[plain]
    \frametitle{Defining Functions}
    \begin{itemize}
        \item[]Little self-contained programs that perform a specific task. 
        %\pause
        \item[] Which you can incorporate into your own, larger programs.
        \newline
        %\pause
        \item[] 'Calling' a function involves: \\ 
        giving a function input, and it will return a value as output.
        \newline
        %\pause
        \item[] print 'Hello Python Class'
        \newline
        %\pause
        \item[] The keyword def introduces a function definition.
    \end{itemize}
\end{frame}
%% 
\begin{frame}[plain]
    \frametitle{Defining Functions}
    \begin{itemize}
        \item[] function definition 
        \lstinputlisting[language=Python]{scripts/factorial.py}
        \pause
        %\newline 
        \item[] function call  
        \lstinputlisting[language=Python]{scripts/call_factorial.py}
    \end{itemize}
\end{frame}
%% 
\begin{frame}[plain]
    \frametitle{Modules}
    \begin{itemize}
        \item[] A module is a file containing Python definitions and statements. 
        \pause
        \item[] factorial.py:
        \lstinputlisting[language=Python]{scripts/factorial.py}
        \pause
        \item[] Import this module: 
        \lstinputlisting[language=Python]{scripts/import_module.py}
    \end{itemize}
\end{frame}
%%
\begin{frame}[plain]
    \frametitle{Regular Expression Operations}
    \begin{itemize}
        \item[] The module \textbf{\textit{re}} provides full support for regular expressions. 
        \pause
        \item[] Meta characters:
        \lstinputlisting[language=Python]{scripts/meta_char_1.py}
        \pause
        \item[] Predefined characters:
        \lstinputlisting[language=Python]{scripts/meta_char_2.py}
        \pause
        \item[] For further references url: \url{https://docs.python.org/2/library/re.html} 
    \end{itemize}
\end{frame}
%%
\begin{frame}[plain]
    \frametitle{Match Function}
    \begin{itemize}
        \item[] Syntax pattern for match function:
        \newline
        \lstinputlisting[language=Python]{scripts/re_match_1.py} 
        %\newline
        \pause
        \lstinputlisting[language=Python]{scripts/re_match_2.py}
    \end{itemize}
\end{frame}
%%
\begin{frame}[plain]
    \frametitle{Matching vs Searching}
    \begin{itemize}
        \item[] match checks for a match only at the beginning of the string
        \pause
        \item[] search checks for a match anywhere in the string
        \pause
        \newline
        \lstinputlisting[language=Python]{scripts/match_search.py}
    \end{itemize}
\end{frame}
%%
\begin{frame}[plain]
    \frametitle{Search and Replace}
    \begin{itemize}
        \item[] Some of the \textbf{re} methods that use regular expressions is \textbf{sub}.
        \newline
        \pause
        \lstinputlisting[language=Python]{scripts/search_replace_1.py}
        \pause
        \lstinputlisting[language=Python]{scripts/search_replace_2.py}
    \end{itemize}
\end{frame}
%%
\begin{frame}[plain]
    \frametitle{Errors and Exceptions}
    \begin{itemize}
        \item[] Syntax errors.
        \lstinputlisting[language=Python]{scripts/syn_error.py}
        %\newline
        \pause
        \item[] An exception is an event, which occurs during the execution of a program.
        \lstinputlisting[language=Python]{scripts/excp_handle.py}
        %\newline
        \pause
        \item[] For further references url: \url{https://docs.python.org/2/tutorial/errors.html} 
    \end{itemize}
\end{frame}
%%
\begin{frame}[plain]
    \frametitle{The assert Statement}
    \begin{itemize}
        \item[] Python interpreter evaluates the expression,
        \item[] If the expression is false, raises an AssertionError exception.
        \newline
        \lstinputlisting[language=Python]{scripts/assert_1.py}
        \pause
        \item[] Test case to check for valid user input phone numbers: 
        \newline
        \lstinputlisting[language=Python]{scripts/assert_2.py}
    \end{itemize}
\end{frame}
%
\begin{frame}[plain]
    \thankyou
\end{frame}
\end{document}
