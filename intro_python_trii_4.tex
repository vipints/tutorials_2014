\documentclass[11pt]{beamer}
\usepackage[utf8]{inputenc}
%
\mode<presentation> {
\usetheme{CambridgeUS}
}
%
\setbeamercolor{frametitle}{fg=cyan!80!black}
\setbeamercolor{title}{fg=cyan!80!black}
\setbeamercolor{palette tertiary}{fg=black, bg=cyan}
\setbeamercolor{palette primary}{fg=black, bg=gray!30!white}
\setbeamercolor{palette secondary}{fg=black, bg=gray!20!white}
%
\setbeamertemplate{headline}
{
  \leavevmode%
  \hbox{%
%  %\begin{beamercolorbox}[wd=.5\paperwidth,ht=2.25ex,dp=1ex,right]{section in head/foot}%
%  %  \usebeamerfont{section in head/foot}\insertsectionhead\hspace*{2ex}
%  %\end{beamercolorbox}%
%  %\begin{beamercolorbox}[wd=.5\paperwidth,ht=2.25ex,dp=1ex,left]{subsection in head/foot}%
%  %  \usebeamerfont{subsection in head/foot}\hspace*{2ex}\insertsubsectionhead
%  %\end{beamercolorbox}
  }%
  \vskip0pt%
}
\setbeamertemplate{footline}{
  \leavevmode%
  \hbox{%
%  %\begin{beamercolorbox}[wd=.5\paperwidth,ht=2.25ex,dp=1ex,center]{author in head/foot}%
%  %  \usebeamerfont{author in head/foot}{Vipin @ R\"{a}tschLab}
%  %\end{beamercolorbox}%
%  %\begin{beamercolorbox}[wd=.5\paperwidth,ht=2.25ex,dp=1ex,center]{title in head/foot}%
%  %  \usebeamerfont{title in head/foot}\insertshorttitle
%  %\end{beamercolorbox}%
%  %\begin{beamercolorbox}[wd=.333333\paperwidth,ht=2.25ex,dp=1ex,right]{date in head/foot}%
%    %\usebeamerfont{date in head/foot}\insertshortdate{}\hspace*{2em}
%  %\end{beamercolorbox}
  }%
  \vskip0pt%
}
%
\usepackage{graphicx} 
\usepackage{booktabs}
\setbeamertemplate{navigation symbols}{}
\usepackage{listings}
\usepackage{color}

\definecolor{codecomment}{rgb}{152,0,0}
\definecolor{codegray}{rgb}{0.5,0.5,0.5}
\definecolor{codepurple}{rgb}{0.58,0,0.82}
\definecolor{backcolour}{rgb}{0.95,0.95,0.92}
\definecolor{codekeys}{rgb}{30,0,100}

\lstdefinestyle{mystyle}{
    backgroundcolor=\color{backcolour},   
    commentstyle=\color{codecomment},
    keywordstyle=\color{codekeys},
    numberstyle=\tiny\color{codegray},
    stringstyle=\color{codepurple},
    basicstyle=\footnotesize,
    breakatwhitespace=false,         
    breaklines=true,                 
    captionpos=b,                    
    keepspaces=true,                 
    numbers=left,                    
    numbersep=5pt,                  
    showspaces=false,                
    showstringspaces=false,
    showtabs=false,                  
    tabsize=4
}

\lstset{style=mystyle}

\newcommand{\thankyou}
{\begin{center} Would love to hear your experience! \\  gabow@cbio.mskcc.org \\ vipin@cbio.mskcc.org \end{center}}
%
\title[pyClass1]{Introduction to Programming with Python - Day 4}
\author{vipin@cbio.mskcc.org}
%
\institute[cBio@MSKCC]
{
    R{\"a}tsch Laboratory, Computational Biology Center\\
    Memorial Sloan Kettering Cancer Center\\
}
\date{13 November 2014}
%
\begin{document}
\maketitle
%
\begin{frame}[plain]
    \frametitle{Python for Scientific Computing}
    \begin{itemize}
        \item[] SciPy is a Python-based ecosystem of open-source software for science. 
        \newline 
        %\pause 
        \item[] NumPy - Powerful N-dimensional array object 
        \newline
        %\pause
        \item[] Matplotlib - 2D plotting library
        \newline
        %\pause
        \item[] IPython - Interactive python console
        \newline
        %\pause
        \item[] pandas - Data structures
    \end{itemize}
\end{frame}
%
%
\begin{frame}[plain]
    \frametitle{NumPy - multidimensional data arrays}
    \begin{itemize}
        \item[] The \textbf{numpy} package (module) is used in almost all numerical computation using Python.
        \newline
        %\pause
        \item[] Provide high-performance vector, matrix and higher-dimensional data structures for Python.
        \newline
        %\pause 
        \item[] In the \textbf{numpy} package the terminology used for vectors, matrices and higher-dimensional data sets is \textit{array}.
        \newline
        %\pause 
         \lstinputlisting[language=Python]{scripts/import_np_1.py}
         %\pause
         \item[] The recommended convention to import numpy is:
        \newline
         %\pause
         \lstinputlisting[language=Python]{scripts/import_np_2.py}
    \end{itemize}
\end{frame}
%
\begin{frame}[plain]
    \frametitle{NumPy - manual construction of arrays}
    \begin{itemize}
        \item[] 1-Dimension: 
         \lstinputlisting[language=Python]{scripts/np_1d.py}
        %\pause 
        \item[] 2-Dimension: 
         \lstinputlisting[language=Python]{scripts/np_2d.py}
    \end{itemize}
\end{frame}
%
\begin{frame}[plain]
    \frametitle{NumPy - basic data types}
    \begin{itemize}
        \item[] Different data types allows to store data efficiently.
        \newline
        %\pause
         \lstinputlisting[language=Python]{scripts/np_dtype_1.py}
         %\pause
         \item[] ValueError - assign a value of the wrong type to an element
         \lstinputlisting[language=Python]{scripts/np_dtype_2.py}
         %\pause
         \item[] Other data types: 
        \begin{itemize}
            \item[] Complex 
            \item[] Bool 
            \item[] String
        \end{itemize}
         %\pause
        \lstinputlisting[language=Python]{scripts/np_dtype_3.py}
    \end{itemize}
\end{frame}
%
\begin{frame}[plain]
    \frametitle{NumPy - function for creating arrays}
    \begin{itemize}
        \item[] arange
        \lstinputlisting[language=Python]{scripts/np_range.py}
        %\pause
        \item[] linspace
        \lstinputlisting[language=Python]{scripts/np_linspace.py}
        %\pause
        \item[] random
        \lstinputlisting[language=Python]{scripts/np_rand.py}
        %\pause
        \lstinputlisting[language=Python]{scripts/np_randn.py}
    \end{itemize}
\end{frame}
%
\begin{frame}[plain]
    \frametitle{NumPy - common arrays}
    \begin{itemize}
        \item[] zeros  
        \lstinputlisting[language=Python]{scripts/np_zeros.py}
        %\pause
        \item[] ones 
        \lstinputlisting[language=Python]{scripts/np_ones.py}
        %\pause
        \item[] diag  
        \lstinputlisting[language=Python]{scripts/np_diag.py}
        %\pause
        \item[] eye   
        \lstinputlisting[language=Python]{scripts/np_eye.py}
        %\pause
        \item[] More details: \url{http://wiki.scipy.org/Tentative_NumPy_Tutorial}
    \end{itemize}
\end{frame}
%
\begin{frame}[plain]
    \frametitle{NumPy - indexing}
    \begin{itemize}
        \item[] Index elements in an array using the square bracket and indices. 
        \lstinputlisting[language=Python]{scripts/np_idx_1.py}
        %\pause
        \lstinputlisting[language=Python]{scripts/np_idx_2.py}
        %\pause
        \item[] If we omit an index of a multidimensional array it returns the whole row
        \lstinputlisting[language=Python]{scripts/np_idx_3.py}
        %\pause
        \item[] Assigning new values to elements in an array using indexing
        \lstinputlisting[language=Python]{scripts/np_idx_4.py}
    \end{itemize}
\end{frame}
%
\begin{frame}[plain]
    \frametitle{NumPy - slicing}
    \begin{itemize}
        \item[] Index slicing syntax \textbf{xm[lower:upper:step]} to extract part of an array. 
        \lstinputlisting[language=Python]{scripts/np_sl_1.py}
        %\pause
        \item[] We can omit any of the three parameters in \textbf{xm[lower:upper:step]}
        \lstinputlisting[language=Python]{scripts/np_sl_2.py}
        %\pause 
        \item[] Index slicing works exactly the same way for multidimensional arrays.
        \lstinputlisting[language=Python]{scripts/np_sl_3.py}
    \end{itemize}
\end{frame}
%
\begin{frame}[plain]
    \frametitle{Matplotlib - basic visualization}
    \begin{itemize}
        \item[] \textit{Matplotlib} is a 2D plotting package.  
        %\pause
        \lstinputlisting[language=Python]{scripts/plt_1d.py}
        %\pause
        \lstinputlisting[language=Python]{scripts/plt_2d.py}
        %\pause
        \item[] Reference: \url{http://matplotlib.org/}\\
        \url{http://web.stanford.edu/~mwaskom/software/seaborn/}
    \end{itemize}
\end{frame}
%
\begin{frame}[plain]
    \frametitle{BioPython - python modules for bioinformatics}
    \begin{itemize}
        \item[] A set of free Python modules for working with sequence analysis.
        \newline
        %\pause 
        \begin{itemize}
            \item[] AlignIO 
            \item[] SearchIO
            \item[] BioSQL
            \item[] Seq
            \item[] SeqIO
            \item[] SeqRecord
            \newline
            %\pause
        \end{itemize}
        \item[] Details: \url{http://biopython.org}
    \end{itemize}
\end{frame}
%
\begin{frame}[plain]
    \frametitle{Seq - basic sequence tools}
    \begin{itemize}
        \item[]  BioPython represents sequences with the \textbf{Seq} object. 
        \newline
        %\pause
        \lstinputlisting[language=Python]{scripts/seq.py}
    \end{itemize}
\end{frame}
%
\begin{frame}[plain]
    \frametitle{SeqIO - basic fileIO}
    \begin{itemize}
        \item[] \textbf{parse} function can handle GenBank, FastQ, Fasta formats.  
        \newline
        %\pause
        \lstinputlisting[language=Python]{scripts/seqIO.py}
        %\pause
        \lstinputlisting[language=Python]{scripts/seqIOfq.py}
    \end{itemize}
\end{frame}
%
\begin{frame}[plain]
    \frametitle{SeqRecord - handling sequence records}
    \begin{itemize}
        \item[] Hold a sequence (as a Seq object) with identifiers (ID and name).
        %\pause
        \lstinputlisting[language=Python]{scripts/seqrecord.py}
    \end{itemize}
\end{frame}
%
\begin{frame}[plain]
    \frametitle{GFF Parsing - parsing genome annotation file}
    \begin{itemize}
        \item[] Not yet integrated into the core biopython  
        \item[] \url{http://github.com/chapmanb/bcbb/tree/master/gff}  
        \newline
        %\pause
        \item[] Collection of different annotation conveter programs 
        \item[] \url{https://github.com/vipints/GFFtools-GX} 
    \end{itemize}
\end{frame}
%
\begin{frame}[plain]
    \frametitle{\textbf{pip} or \textbf{easy\_install} - managing local python packages}
    \begin{itemize}
        \item[] Machine learning toolbox in python
        \begin{itemize}
            \item []SHOGUN - \url{shogun-toolbox.org}
            \item []scikit-learn - \url{scikit-learn.org} 
        %\pause
        \end{itemize}
        \lstinputlisting[language=Bash]{scripts/pip_install.sh}
    \end{itemize}
\end{frame}
%
\begin{frame}[plain]
    \frametitle{\textbf{cPickle} - data persistence}
    \begin{itemize}
        \item[] Methods \textbf{dump()}, \textbf{load()} 
        \newline
        %\pause
        \lstinputlisting[language=Python]{scripts/cpickle_dump.py}
        %\pause
        \lstinputlisting[language=Python]{scripts/cpickle_load.py}
    \end{itemize}
\end{frame}
%
\begin{frame}[plain]
    \frametitle{\textbf{Django} - python web framework}
    \begin{itemize}
        \item[] Creating a project: django-admin.py startproject mysite 
        \newline
        %\pause 
        \item[] Add names to \textit{mysite settings.py} \textit{mysite urls.py} 
        %\pause 
        \item[] python manage.py startapp app\_name
        %\pause 
        \item[] Edit view.py 
        \lstinputlisting[language=Python]{scripts/views.py}
        %\pause 
        \item[] python manage.py runserver
        \item[] Details: \url{https://www.djangoproject.com/}
    \end{itemize}
\end{frame}
%
\begin{frame}[plain]
    \thankyou
\end{frame}
\end{document}
